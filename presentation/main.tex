%%%%%%%%%%%%%%%%%%%%%%%%%%%%%%%%%%%%%%%%%%%%%%%%%%%%%%
% A Beamer template for HKUST (GZ)                   %
% Based on THU beamer theme                          %
% Author: Yuxuan HU                                  %
% Date: Aug 2024                                    %
% LPPL Licensed.                                     %
%%%%%%%%%%%%%%%%%%%%%%%%%%%%%%%%%%%%%%%%%%%%%%%%%%%%%%

\documentclass[serif, aspectratio=169]{beamer}
%\documentclass[serif]{beamer}  % for 4:3 ratio
\usepackage[T1]{fontenc} 
\usepackage{fourier} % see "http://faq.ktug.org/wiki/uploads/MathFonts.pdf" for other options
\usepackage{hyperref}
\usepackage{latexsym,amsmath,xcolor,multicol,booktabs,calligra}
\usepackage{graphicx,pstricks,listings,stackengine}
\usepackage{lipsum}
\usepackage[dvipsnames]{xcolor}


\author{Francesco Braicovich, Nikhil Joseph, Qin Zhihua (Ivan)}
\title{Volatility Modeling \& Trading}
\subtitle{Predicting Realized Volatility Across Multiple Horizons}
\institute{
   Statistical Modelling in Financial Engineering - IEDA 4000E \\
    The Hong Kong University of Science and Technology \\
}
\date{\small November 2025}
\usepackage{HKUSTstyle}

% Reduce spacing between bullet points and fix overflow issues
\setlength{\itemsep}{1pt}
\let\olditemize\itemize
\renewcommand{\itemize}{\olditemize\setlength{\itemsep}{1pt}\setlength{\parskip}{0pt}}

% Reduce spacing in blocks and columns
\setbeamertemplate{block begin}{
  \vskip.5ex
  \begin{beamercolorbox}[ht=2.4ex,dp=1ex,center,rounded=true,shadow=true,leftskip=1ex,colsep*=.75ex]{block title}%
    \usebeamerfont*{block title}\insertblocktitle
  \end{beamercolorbox}%
  {\ifbeamercolorempty[bg]{block body}{}{\nointerlineskip\vskip-0.5pt}}%
  \usebeamerfont{block body}%
  \begin{beamercolorbox}[rounded=true,shadow=true,leftskip=1ex,colsep*=.75ex,sep=1ex,vmode]{block body}%
    \ifbeamercolorempty[bg]{block body}{\vskip-.25ex}{\vskip-.75ex}\vbox{}%
}
\setbeamertemplate{block end}{
  \end{beamercolorbox}
}

% Reduce font size globally for better fit
\setbeamerfont{itemize/enumerate body}{size=\small}
\setbeamerfont{itemize/enumerate subbody}{size=\footnotesize}

% defs
\def\cmd#1{\texttt{\color{red}\footnotesize $\backslash$#1}}
\def\env#1{\texttt{\color{blue}\footnotesize #1}}
% set colors
\definecolor{hkustyellow}{RGB}{167, 131, 55}
\definecolor{hkustblue}{RGB}{0, 56, 116}
\definecolor{hkustred}{RGB}{209, 51, 59}


\lstset{
    basicstyle=\ttfamily\small,
    keywordstyle=\bfseries\color{deepblue},
    emphstyle=\ttfamily\color{deepred},    % Custom highlighting style
    stringstyle=\color{deepgreen},
    numbers=left,
    numberstyle=\small\color{halfgray},
    rulesepcolor=\color{red!20!green!20!blue!20},
    frame=shadowbox,
}

%- --- --- --- --- --- --- --- --- --- --- --- --- --- --- --- 
\begin{document}

\begin{frame}
    \titlepage
    \vspace*{-0.6cm}
    \begin{figure}[htpb]
        \begin{center}
            \includegraphics[keepaspectratio, scale=0.02]{pic/UST.png}
        \end{center}
    \end{figure}
\end{frame}

\begin{frame}    
\tableofcontents[sectionstyle=show,
subsectionstyle=show/shaded/hide,
subsubsectionstyle=show/shaded/hide]
\end{frame}

% ================================================================
% SECTION 1: INTRODUCTION & PROJECT OVERVIEW
% ================================================================

\section{Introduction}

\begin{frame}{Research Motivation}
        Can we predict future realized volatility accurately enough to identify mispricings in the volatility market?
    
    \vspace{0.2cm}
    
    \begin{columns}[T]
        \begin{column}{0.48\textwidth}
            \textbf{The Volatility Risk Premium (VRP):}
            \begin{itemize}
                \item VIX (Implied Volatility) typically exceeds Realized Volatility
                \item Investors pay premium for protection
                \item Systematic opportunity for sellers
            \end{itemize}
        \end{column}
        
        \begin{column}{0.48\textwidth}
            \textbf{Our Approach:}
            \begin{itemize}
                \item Forecast RV using multiple models
                \item Compare to VIX (market expectations)
                \item Identify when VIX is overpriced
                \item $\text{VRP} = \text{VIX} - \widehat{RV}$
            \end{itemize}
        \end{column}
    \end{columns}
\end{frame}

\begin{frame}{Project Scope}
        We predict realized volatility across \textbf{four time horizons}:
        \begin{itemize}
            \item h=2 days (ultra-short term)
            \item h=5 days (one trading week)
            \item h=10 days (two weeks)
            \item h=30 days (one month, matches VIX)
        \end{itemize}
    
    \vspace{0.2cm}
    
        Three distinct modeling philosophies:
        \begin{enumerate}
            \item \textbf{GARCH/EGARCH}: Econometric approach (leverage effect, volatility clustering)
            \item \textbf{LSTM-RV}: Deep learning on historical realized volatility
            \item \textbf{LSTM-VIX}: Deep learning to "de-bias" implied volatility
        \end{enumerate}
\end{frame}

\begin{frame}{Test Period \& Data Sources}
    \begin{columns}[T]
        \begin{column}{0.58\textwidth}
            \textbf{Data Coverage (1990-2025):}
            \begin{itemize}
                \item Train: 1993-2015 (23 years)
                \item Validation: 2016-2019 (4 years)
                \item Test: 2020-2025 (5.5 years)
                \item Daily frequency, Yahoo Finance
            \end{itemize}
            
            \vspace{0.2cm}
            
            \textbf{Instruments:}
            \begin{itemize}
                \item SPY: S\&P 500 ETF
                \item \^{}VIX: CBOE Volatility Index
            \end{itemize}
        \end{column}
        
        \begin{column}{0.38\textwidth}
            \textbf{Why This Period?}
            \begin{itemize}
                \item Dot-com (2000-2002)
                \item Financial crisis (2008)
                \item COVID crash (2020)
                \item Inflation (2022)
                \item Multiple regimes
            \end{itemize}
            
            \vspace{0.2cm}
            
            \textcolor{hkustred}{\textbf{Rigorous out-of-sample testing!}}
        \end{column}
    \end{columns}
\end{frame}

% ================================================================
% SECTION 2: DATA METHODOLOGY
% ================================================================

\section{Data Methodology}

\begin{frame}{Feature Engineering: Target Variable}
        $$RV_{t,h}^{fwd} = \sqrt{252} \times \text{std}(r_{t+1}, r_{t+2}, \ldots, r_{t+h})$$
    
    \vspace{0.2cm}
    
    \begin{columns}[T]
        \begin{column}{0.48\textwidth}
            \textbf{Components:}
            \begin{itemize}
                \item $r_{t+i}$: Log return on day $i$
                \item $\text{std}(\cdot)$: Standard deviation
                \item $\sqrt{252}$: Annualization factor
                \item $h$: Forecast horizon
            \end{itemize}
        \end{column}
        
        \begin{column}{0.48\textwidth}
            \textbf{Key Properties:}
            \begin{itemize}
                \item Uses \textbf{only future returns} ($t+1$ to $t+h$)
                \item Zero lookahead bias by construction
                \item Computed for all h $\in$ \{2, 5, 10, 30\}
                \item Represents actual volatility realized
            \end{itemize}
        \end{column}
    \end{columns}
    
    \vspace{0.2cm}
    
    \begin{center}
        \textcolor{hkustblue}{\textbf{This is what the market tries to predict with VIX}}
    \end{center}
\end{frame}

\begin{frame}{Feature Engineering: Input Features}
        Models can only use information available \textbf{up to and including} time $t$
    
    \vspace{0.2cm}
    
    \begin{enumerate}
        \item \textbf{Backward Realized Volatility} ($RV_{t,h}^{back}$)
        \begin{itemize}
            \item Historical volatility over \textit{past} $h$ days
            \item $RV_{t,h}^{back} = \sqrt{252} \times \text{std}(r_{t-h+1}, \ldots, r_t)$
            \item Captures recent market turbulence
            \item Used by LSTM-RV models
        \end{itemize}
        
        \vspace{0.2cm}
        
        \item \textbf{Implied Volatility} ($IV_t$ from VIX)
        \begin{itemize}
            \item Market's expectation of future 30-day volatility
            \item Forward-looking collective wisdom
            \item Already available before market close at $t$
            \item Used by LSTM-VIX model
        \end{itemize}
    \end{enumerate}
\end{frame}

\begin{frame}{Log Transformation: Model-Specific Approach}
        Volatility is strictly positive but unbounded $\Rightarrow$ skewed, heteroskedastic
    
    \vspace{0.2cm}
    
    \begin{columns}[T]
        \begin{column}{0.48\textwidth}
            \textbf{Before Log Transform:}
            \begin{itemize}
                \item Distribution highly skewed
                \item Extreme outliers (crisis periods)
                \item Models struggle with scale
                \item Predictions can be negative!
            \end{itemize}
            
            \vspace{0.2cm}
            
            $$y = RV \in (0, \infty)$$
        \end{column}
        
        \begin{column}{0.48\textwidth}
            \textbf{After Log Transform:}
            \begin{itemize}
                \item Approximately Normal distribution
                \item Compressed extreme values
                \item Stabilized variance
                \item Predictions always positive after $\exp(\cdot)$
            \end{itemize}
            
            \vspace{0.2cm}
            
            $$y = \log(RV) \in (-\infty, \infty)$$
        \end{column}
    \end{columns}
    
    \vspace{0.15cm}
    
    \begin{itemize}
        \item \textbf{LSTM Models:} Predict $\log(RV)$ directly, then apply $\exp(\cdot)$ to get $RV$
        \item \textbf{GARCH/EGARCH:} Models $\log(\sigma^2_t)$ internally; predicts $RV$ from variance forecasts (not log-transformed)
    \end{itemize}
\end{frame}

\begin{frame}{Lookahead Bias Prevention: Critical Detail}
        At time $t$, if we predict $RV_{fwd}[t]$ using returns from $t+1$ to $t+h$, and our training set ends exactly at validation start, the \textbf{last training samples leak into validation period!}
    
    \vspace{0.2cm}
    
        \textbf{Training Gap:} Remove samples where target horizon extends into validation
        \begin{itemize}
            \item Training ends: $\text{train\_end} - (h \times 1.5)$ calendar days
            \item Factor 1.5 accounts for weekends/holidays
            \item For h=30: $\sim$45 calendar day gap = 30 trading days safe
        \end{itemize}
    
    \vspace{0.2cm}
    
    \begin{center}
        \textcolor{hkustblue}{\textbf{Result: Zero information leakage from future to past}}
    \end{center}
\end{frame}

\begin{frame}{Market Overview: 35 Years of Data}
    \begin{columns}[T]
        \begin{column}{0.5\textwidth}
            \textbf{Key Observations:}
            
            \vspace{0.2cm}
            
            \textbf{SPY (Top Panel):}
            \begin{itemize}
                \item Strong upward trend
                \item Non-stationary process
                \item Multiple bull/bear cycles
                \item Returns, not prices, are predictable
            \end{itemize}
            
            \vspace{0.2cm}
            
            \textbf{VIX (Bottom Panel):}
            \begin{itemize}
                \item Mean-reverting around 15-20
                \item Sharp spikes during crises
                \item Dot-com (2000-2002): VIX $\sim$45
                \item Financial crisis (2008): VIX $>$80
                \item COVID (2020): VIX $\sim$80
            \end{itemize}
        \end{column}
        
        \begin{column}{0.5\textwidth}
            \begin{figure}
                \centering
                \includegraphics[width=\textwidth]{pic/01_market_overview.png}
            \end{figure}
        \end{column}
    \end{columns}
    
    \vspace{0.2cm}
    
    \begin{center}
        \textcolor{hkustred}{\textbf{Inverse relationship: VIX explodes when SPY crashes}}
    \end{center}
\end{frame}


\begin{frame}{Autocorrelation Analysis}
    \begin{columns}[T]
        \begin{column}{0.45\textwidth}
            
            \vspace{0.2cm}
            
            \textbf{Raw Returns:}
            \begin{itemize}
                \item ACF stays within blue bands
                \item No significant autocorrelation
                \item \textcolor{hkustblue}{Returns are unpredictable}
                \item Efficient Market Hypothesis holds
                \item Cannot predict next day's return
            \end{itemize}
        \end{column}
        
        \begin{column}{0.55\textwidth}
            \vspace{0.2cm}
            \begin{figure}
                \centering
                \includegraphics[width=1\textwidth]{pic/stationarity_acf.png}
            \end{figure}
        \end{column}
    \end{columns}

\end{frame}


\begin{frame}{Statistical Test 1: Stationarity (ADF Test)}
        Non-stationary series have time-varying mean/variance $\Rightarrow$ \textbf{spurious regressions}
    
    \vspace{0.2cm}
    
    \begin{columns}[T]
        \begin{column}{0.48\textwidth}
            \textbf{Augmented Dickey-Fuller Test:}
            \begin{itemize}
                \item $H_0$: Unit root (non-stationary)
                \item $H_1$: Stationary
                \item Test statistic: -58.34
                \item \textbf{p-value: $<$ 0.0001}
            \end{itemize}
            
            \vspace{0.2cm}
            
            $$\Delta r_t = \alpha + \beta r_{t-1} + \sum_{i=1}^p \gamma_i \Delta r_{t-i} + \epsilon_t$$
            
            \vspace{0.2cm}
            
            If $\beta < 0$ significantly $\Rightarrow$ mean reversion
        \end{column}
        
        \begin{column}{0.48\textwidth}
            \textbf{Result Interpretation:}
            \begin{itemize}
                \item \textcolor{hkustblue}{Reject $H_0$ at 0.1\% level}
                \item Returns are stationary
                \item Mean and variance are stable over time
                \item Safe to use standard regression models
            \end{itemize}
            
            \vspace{0.2cm}
            
            \textbf{Why This Test:}
            \begin{itemize}
                \item Prices grow exponentially (non-stationary)
                \item Returns fluctuate around zero (stationary)
                \item We model returns, not prices
                \item Avoids spurious correlations
            \end{itemize}
        \end{column}
    \end{columns}
\end{frame}


\begin{frame}{Statistical Test 2: Stationarity (KPSS Test)}
        KPSS has opposite null hypothesis: $H_0$ = stationary (vs ADF: $H_0$ = non-stationary)
    
    \vspace{0.2cm}
    
    \begin{columns}[T]
        \begin{column}{0.48\textwidth}
            \textbf{Test Results:}
            \begin{itemize}
                \item Test statistic: 0.116
                \item Critical (5\%): 0.463
                \item \textbf{p-value: 0.10}
                \item \textcolor{hkustblue}{Fail to reject $H_0$}
            \end{itemize}
            
            \vspace{0.2cm}
            
            Tests: $r_t = \mu + \eta_t$ (stationary)
            
            vs trend-stationary or random walk
        \end{column}
        
        \begin{column}{0.48\textwidth}
            \textbf{Interpretation:}
            \begin{itemize}
                \item Evidence for stationarity
                \item Consistent with ADF
                \item Robust conclusion from both tests
            \end{itemize}
            
            \vspace{0.2cm}
            
            \textbf{Why Dual Testing:}
            \begin{itemize}
                \item ADF: reject non-stationarity
                \item KPSS: accept stationarity
                \item Agreement $\Rightarrow$ high confidence
            \end{itemize}
        \end{column}
    \end{columns}
\end{frame}

\begin{frame}{Visual Evidence: Volatility Clustering}
    \begin{columns}[T]
        \begin{column}{0.48\textwidth}
            
            \textbf{Calm Periods:}
            \begin{itemize}
                \item 1993-1997: Low, stable volatility
                \item 2003-2007: "Great Moderation"
                \item 2012-2019: Extended calm
                \item Returns within $\pm$1\% daily
            \end{itemize}
            
            \vspace{0.2cm}
            
            \textbf{Crisis Periods:}
            \begin{itemize}
                \item 2008-2009: Persistent wild swings
                \item March 2020: COVID shock
                \item 2022: Inflation fears
                \item Returns $\pm$5-10\% daily for weeks
            \end{itemize}
        \end{column}
        
        \begin{column}{0.48\textwidth}
            \begin{figure}
                \centering
                \includegraphics[width=\textwidth]{pic/02_volatility_clustering.png}
            \end{figure}
        \end{column}
    \end{columns}
    
    \vspace{0.2cm}
    
    \begin{center}
        \textcolor{hkustblue}{\textbf{Volatility has "regimes" that persist - this is why GARCH works}}
    \end{center}
\end{frame}

\begin{frame}{Autocorrelation Analysis}
        \textit{"Large changes tend to be followed by large changes, of either sign, and small changes tend to be followed by small changes"} - Benoit Mandelbrot
    
    \vspace{0.2cm}
    \begin{columns}[T]
        \begin{column}{0.45\textwidth}
            \vspace{0.35cm}
            \textbf{Squared Returns (Right Panel):}
            \begin{itemize}
                \item ACF far above blue bands
                \item Strong autocorrelation up to 20 lags
                \item \textcolor{hkustred}{Volatility is predictable!}
                \item Today's volatility $\Rightarrow$ tomorrow's
                \item Justifies volatility modeling
            \end{itemize}

        \end{column}
        
        \begin{column}{0.45\textwidth}
            \begin{figure}
                \centering
                \includegraphics[width=1\textwidth]{pic/arch_acf.png}
            \end{figure}
        \end{column}
    \end{columns}

\end{frame}

\begin{frame}{Statistical Test 3: Volatility Clustering (ARCH-LM)}
    
    \begin{columns}[T]
        \begin{column}{0.48\textwidth}
            \textbf{ARCH-LM Test:}
            \begin{itemize}
                \item $H_0$: No ARCH effects (homoskedastic)
                \item $H_1$: ARCH effects present
                \item Test statistic: 1389.1
                \item \textbf{p-value: $<$ 0.0001}
            \end{itemize}
            
            \vspace{0.2cm}
            
            Tests if squared residuals are autocorrelated:
            $$\epsilon_t^2 = \alpha_0 + \sum_{i=1}^q \alpha_i \epsilon_{t-i}^2 + u_t$$
            
            If $\alpha_i$ significant $\Rightarrow$ volatility clustering
        \end{column}
        
        \begin{column}{0.48\textwidth}
            \textbf{Result Interpretation:}
            \begin{itemize}
                \item \textcolor{hkustred}{Strongly reject $H_0$}
                \item Massive test statistic (1389!)
                \item Volatility is \textbf{highly clustered}
                \item Today's volatility predicts tomorrow's
            \end{itemize}
            
            \vspace{0.2cm}
            
            \textbf{Implication for Modeling:}
            \begin{itemize}
                \item Cannot use constant variance models
                \item \textbf{GARCH family is justified}
                \item Volatility has memory
                \item Crisis periods persist for weeks
            \end{itemize}
        \end{column}
    \end{columns}
\end{frame}

\begin{frame}{Visual Evidence: Fat Tails vs Normal Distribution}
    \begin{columns}[T]
        \begin{column}{0.52\textwidth}
            \textbf{Key Findings:}
            
            \vspace{0.2cm}
            
            \begin{itemize}
                \item Gray: Actual distribution
                \item Red: Normal fit
                \item \textcolor{hkustred}{\textbf{Massive tail divergence}}
            \end{itemize}
            
            \vspace{0.2cm}
            
            \textbf{Extreme Events (|return| $>$ 3\%):}
            \begin{itemize}
                \item Normal predicts: $\sim$30 days
                \item Reality: $\sim$180 days
                \item \textbf{6x more frequent!}
            \end{itemize}
            
        \end{column}
        
        \begin{column}{0.44\textwidth}
            \begin{figure}
                \centering
                \includegraphics[width=\textwidth]{pic/03_distribution_fat_tails.png}
            \end{figure}
        \end{column}
    \end{columns}
\end{frame}

\begin{frame}{Q-Q Plot: Visual Test of Normality}
    \begin{columns}[T]
        \begin{column}{0.52\textwidth}
            
            \begin{itemize}
                \item \textbf{S-shaped curve}
                \item Tails deviate from line
                \item \textcolor{hkustred}{Fat tails confirmed}
            \end{itemize}
            
            \vspace{0.5cm}
            
            \textbf{Implication:}
            \begin{itemize}
                \item Extreme events more frequent
                \item Student-t fits better ($\nu \sim 6$)
                \item GARCH must use Student-t
            \end{itemize}
        \end{column}
        
        \begin{column}{0.44\textwidth}
            \begin{figure}
                \centering
                \includegraphics[width=\textwidth]{pic/09_qq_plot.png}
            \end{figure}
        \end{column}
    \end{columns}
\end{frame}


\begin{frame}{Statistical Test 4: Normality (Jarque-Bera)}
        Gaussian assumption affects risk estimates and confidence intervals
    
    \vspace{0.2cm}
    
    \begin{columns}[T]
        \begin{column}{0.48\textwidth}
            \textbf{Test Results:}
            \begin{itemize}
                \item $H_0$: Normal distribution
                \item \textbf{p-value: $<$ 0.0001}
                \item \textcolor{hkustred}{Strongly reject}
            \end{itemize}
            
            \vspace{0.2cm}
            
            $$JB = \frac{n}{6}\left(S^2 + \frac{(K-3)^2}{4}\right)$$
            
            \begin{itemize}
                \item Skewness: -0.11
                \item Excess kurtosis: 10.02
            \end{itemize}
        \end{column}
        
        \begin{column}{0.48\textwidth}
            \textbf{Findings:}
            \begin{itemize}
                \item Slight negative skew
                \item \textbf{Massive fat tails}
                \item Extremes $\gg$ Normal predicts
            \end{itemize}
            
            \vspace{0.2cm}
            
            \textbf{Impact:}
            \begin{itemize}
                \item 3$\sigma$ events: 0.3\% (Normal) vs 2\% (Actual)
                \item \textbf{Crashes 6x more frequent!}
                \item Must use Student-t in GARCH
            \end{itemize}
        \end{column}
    \end{columns}
\end{frame}


\begin{frame}{The Leverage Effect: Why Markets Panic Downward}
    \begin{columns}[T]
        \begin{column}{0.48\textwidth}
            
            \vspace{0.5cm}
            
            \textbf{The Chart Shows:}
            \begin{itemize}
                \item Correlation between today's return and future volatility
                \item X-axis: Days ahead (lag)
                \item Y-axis: Correlation strength
                \item \textcolor{hkustred}{All bars negative!}
            \end{itemize}
            
            \vspace{0.5cm}
            
            \textbf{Key Insight:}
            \begin{itemize}
                \item Negative returns today $\Rightarrow$ high volatility tomorrow
                \item Positive returns today $\Rightarrow$ low volatility tomorrow
                \item Effect persists for 10+ days
            \end{itemize}
            
            \vspace{0.5cm}
            
            \textbf{Why This Matters:}
            \begin{itemize}
                \item Standard GARCH is symmetric
                \item \textcolor{hkustblue}{EGARCH models this asymmetry}
            \end{itemize}
        \end{column}
        
        \begin{column}{0.55\textwidth}
            \vspace{0.5cm}
            \begin{figure}
                \centering
                \includegraphics[width=\textwidth]{pic/05_leverage_effect.png}
            \end{figure}
        \end{column}
    \end{columns}
\end{frame}

\begin{frame}{Feature Correlation Matrix: The Information Landscape}
    \begin{columns}[T]
        \begin{column}{0.52\textwidth}
            \textbf{Strong Correlations ($\rho > 0.75$):}
            \begin{itemize}
                \item VIX $\leftrightarrow$ RV$_{30}$: 0.82
                \item VIX $\leftrightarrow$ RV$_{10}$: 0.79
                \item RV horizons: $>$0.9 (multicollinear)
                \item \textcolor{hkustblue}{IV captures future RV well}
            \end{itemize}
            
            \vspace{0.05cm}
            
            \textbf{Negative ($\rho \sim -0.4$):}
            \begin{itemize}
                \item Returns $\leftrightarrow$ All volatilities
                \item \textcolor{hkustred}{Leverage effect confirmed}
            \end{itemize}
            
            \vspace{0.05cm}

            
            \textbf{Implications:}
            \begin{itemize}
                \item VIX highly informative (67\% R²)
                \item 33\% variance unexplained $\Rightarrow$ opportunity
                \item Returns add asymmetry signal
            \end{itemize}
        \end{column}
        
        \begin{column}{0.55\textwidth}
            \begin{figure}
                \centering
                \includegraphics[width=\textwidth]{pic/08_correlation_matrix.png}
            \end{figure}
        \end{column}
    \end{columns}
\end{frame}


% ================================================================
% SECTION 3: MODEL SELECTION & GARCH
% ================================================================

\section{Model Selection}

\begin{frame}{Model Selection Strategy: Grid Search}
        We test multiple specifications systematically:
        \begin{itemize}
            \item \textbf{Model families}: GARCH vs EGARCH
            \item \textbf{Orders}: $(p, q) \in \{(1,1), (1,2), (2,1), (2,2)\}$
            \item \textbf{Distributions}: Normal, Student-t, Skewed Student-t
        \end{itemize}
    
    \vspace{0.2cm}
    
        \textbf{Two-stage process:}
        \begin{enumerate}
            \item Filter: Pass ARCH-LM test on residuals (no remaining heteroskedasticity)
            \item Rank: Choose lowest BIC among passing models
        \end{enumerate}
    
    \vspace{0.2cm}
    
    \begin{center}
        \textcolor{hkustblue}{\textbf{Winner: EGARCH(2,1) with Skewed Student-t}}
    \end{center}
\end{frame}

\begin{frame}{GARCH: The Symmetric Baseline}
        $$\sigma_t^2 = \omega + \sum_{i=1}^{p} \alpha_i \epsilon_{t-i}^2 + \sum_{j=1}^{q} \beta_j \sigma_{t-j}^2$$
    
    \vspace{0.2cm}
    
    \begin{columns}[T]
        \begin{column}{0.48\textwidth}
            \textbf{Components:}
            \begin{itemize}
                \item $\omega$: Long-run average
                \item $\alpha \epsilon_{t-1}^2$: Shock term
                \item $\beta \sigma_{t-1}^2$: Persistence
            \end{itemize}
        \end{column}
        
        \begin{column}{0.48\textwidth}
            \textbf{Interpretation:}
            \begin{itemize}
                \item Volatility clustering
                \item Mean reversion
                \item \textcolor{hkustred}{Symmetric response}
            \end{itemize}
        \end{column}
    \end{columns}
    
    \vspace{0.2cm}
    
        Squaring $\epsilon_{t-1}$ means +5\% rally and -5\% crash have \textbf{identical} impact on volatility. This contradicts empirical evidence!
\end{frame}

\begin{frame}{EGARCH: Capturing Asymmetry}
        $$\ln(\sigma_t^2) = \omega + \beta \ln(\sigma_{t-1}^2) + \alpha \left( \left| \frac{\epsilon_{t-1}}{\sigma_{t-1}} \right| - \sqrt{\frac{2}{\pi}} \right) + \gamma \frac{\epsilon_{t-1}}{\sigma_{t-1}}$$
    
    \vspace{0.2cm}
    
    \begin{columns}[T]
        \begin{column}{0.58\textwidth}
            \textbf{Key Innovation: The $\gamma$ term}
            \begin{itemize}
                \item When $\gamma < 0$ (typical): negative returns $\Rightarrow$ higher volatility
                \item Captures "panic" vs "greed" asymmetry
                \item Leverage effect built into model
            \end{itemize}
            
            \vspace{0.2cm}
            
            \textbf{Log Specification Benefits:}
            \begin{itemize}
                \item Guarantees $\sigma^2 > 0$ always
                \item No parameter constraints needed
            \end{itemize}
        \end{column}
        
        \begin{column}{0.38\textwidth}
            \textbf{Our Results:}
            \begin{itemize}
                \item EGARCH(2,1) + skewt
                \item BIC: 15,401.9
                \item vs GARCH(2,1) + skewt
                \item BIC: 15,638.5
            \end{itemize}
            
            \vspace{0.2cm}
            
            \textcolor{hkustblue}{\textbf{EGARCH wins by 237 BIC points}}
        \end{column}
    \end{columns}
\end{frame}

\begin{frame}{GARCH vs EGARCH: News Impact Curve}
    \begin{figure}
        \centering
        \includegraphics[width=0.75\textwidth]{pic/10_garch_vs_egarch_nic.png}
    \end{figure}
    \vspace{-0.3cm}
    \begin{itemize}
        \item \textbf{GARCH (blue dashed)}: Symmetric parabola - same response to $\pm$5\% moves
        \item \textbf{EGARCH (red solid)}: Asymmetric - negative shocks increase volatility much more
        \item EGARCH correctly captures market panic dynamics
    \end{itemize}
\end{frame}

\begin{frame}{GARCH Model Diagnostics}
    \begin{columns}[T]
        \begin{column}{0.48\textwidth}
            \textbf{Residual Tests (EGARCH(2,1)):}
            \begin{itemize}
                \item ARCH-LM test: $p = 0.839$
                \item Ljung-Box: $p = 0.851$
                \item \textcolor{hkustblue}{No remaining heteroskedasticity}
            \end{itemize}
            
            \vspace{0.2cm}
            
            \textbf{Parameter Estimates:}
            \begin{itemize}
                \item All coefficients significant
                \item $\gamma < 0$ confirms leverage effect
                \item Persistence: $\alpha + \beta \approx 0.98$
                \item High persistence $\Rightarrow$ shocks last
            \end{itemize}
        \end{column}
        
        \begin{column}{0.48\textwidth}
            \textbf{Skewed Student-t Parameters:}
            \begin{itemize}
                \item Degrees of freedom: $\nu \approx 6$
                \item Skewness parameter: negative
                \item Captures both fat tails and asymmetry
            \end{itemize}
            
            \vspace{0.2cm}
            
            \textbf{Model Quality:}
            \begin{itemize}
                \item BIC: 15,401.9 (lowest)
                \item AIC: 15,358.6
                \item Log-likelihood: -7,669.3
                \item Converged successfully
            \end{itemize}
        \end{column}
    \end{columns}
\end{frame}

% ================================================================
% SECTION 4: DEEP LEARNING MODELS
% ================================================================

\section{Deep Learning Models}

\begin{frame}{Why Neural Networks for Volatility?}
        \begin{itemize}
            \item Assumes volatility follows strict parametric formula
            \item Limited flexibility in capturing complex patterns
            \item May miss regime changes or structural breaks
        \end{itemize}
    
    \vspace{0.2cm}
    
        \begin{itemize}
            \item \textbf{Non-parametric}: Learn patterns directly from data
            \item \textbf{Flexible}: Capture non-linear relationships
            \item \textbf{Adaptive}: Can learn regime-dependent behavior
            \item \textbf{High-dimensional}: Handle multiple features naturally
        \end{itemize}
    
    \vspace{0.2cm}
    
    \begin{center}
        \textcolor{hkustblue}{\textbf{But: Need sequential memory for time series $\Rightarrow$ LSTMs}}
    \end{center}
\end{frame}

\begin{frame}{Recurrent Neural Networks: The Memory Concept}
        \begin{itemize}
            \item \textbf{Feed-forward NN}: Sees each day independently
            \item \textbf{RNN}: Maintains "hidden state" $h_t$ that summarizes history
        \end{itemize}
    
    $$h_t = \tanh(W_x x_t + W_h h_{t-1} + b)$$
    
    \vspace{0.2cm}
    
    \begin{columns}[T]
        \begin{column}{0.48\textwidth}
            \textbf{Intuition:}
            \begin{itemize}
                \item Processes data sequentially
                \item $h_t$ = compressed memory
                \item Can "remember" past patterns
            \end{itemize}
        \end{column}
        
        \begin{column}{0.48\textwidth}
            \textbf{Problem:}
            \begin{itemize}
                \item Vanishing gradients
                \item Forgets distant past
                \item "Goldfish memory"
            \end{itemize}
        \end{column}
    \end{columns}
    
    \vspace{0.2cm}
    
    \begin{center}
        \textcolor{hkustred}{\textbf{Solution: Long Short-Term Memory (LSTM)}}
    \end{center}
\end{frame}

\begin{frame}{LSTM Architecture: Gating Mechanisms}
        LSTMs add a "cell state" $C_t$ with three gates controlling information flow:
    
    \begin{columns}[T]
        \begin{column}{0.3\textwidth}
            \textbf{Forget Gate:}
            \begin{itemize}
                \item What to discard
                \item $f_t = \sigma(W_f \cdot [h_{t-1}, x_t])$
            \end{itemize}
        \end{column}
        
        \begin{column}{0.3\textwidth}
            \textbf{Input Gate:}
            \begin{itemize}
                \item What to store
                \item $i_t = \sigma(W_i \cdot [h_{t-1}, x_t])$
            \end{itemize}
        \end{column}
        
        \begin{column}{0.3\textwidth}
            \textbf{Output Gate:}
            \begin{itemize}
                \item What to output
                \item $o_t = \sigma(W_o \cdot [h_{t-1}, x_t])$
            \end{itemize}
        \end{column}
    \end{columns}
    
    \vspace{0.2cm}
    
        Can maintain long-term memory (e.g., "we're in a crisis regime") over 60+ days without degradation
\end{frame}

\begin{frame}{Our LSTM Architectures}
    \begin{columns}[T]
        \begin{column}{0.48\textwidth}
            \textbf{LSTM-RV (The Historian)}
            
                \begin{itemize}
                    \item Input: Past RV (backward-looking)
                    \item Bidirectional LSTM
                    \item 2 layers, 128 units each
                    \item Dropout: 0.2
                \end{itemize}
            
            \textbf{Role:}
            \begin{itemize}
                \item Captures complex autoregressive patterns
                \item Learns non-linear volatility persistence
                \item Identifies regime transitions
            \end{itemize}
        \end{column}
        
        \begin{column}{0.48\textwidth}
            \textbf{LSTM-VIX (The Translator)}
            
                \begin{itemize}
                    \item Input: VIX (implied volatility)
                    \item Bidirectional LSTM
                    \item 2 layers, 128 units each
                    \item Dropout: 0.2
                \end{itemize}
            
            \textbf{Role:}
            \begin{itemize}
                \item "De-biases" VIX
                \item Learns $f(\text{VIX}) \to \text{RV}$
                \item Forward-looking (market expectations)
            \end{itemize}
        \end{column}
    \end{columns}
    
    \vspace{0.2cm}
    
    \begin{center}
        \textcolor{hkustblue}{\textbf{Both use 60-day sequence length for context}}
    \end{center}
\end{frame}

\begin{frame}{Bidirectional LSTMs: Reading History Both Ways}
        In time series, future context (within training window) can help understand patterns:
    
    \vspace{0.2cm}
    
    \begin{columns}[T]
        \begin{column}{0.48\textwidth}
            \textbf{Forward LSTM:}
            \begin{itemize}
                \item Reads $t-60 \to t-1 \to t$
                \item Captures trends
                \item Sees how we got here
            \end{itemize}
            
            \vspace{0.2cm}
            
            \textbf{Combined Output:}
            \begin{itemize}
                \item Concatenate both directions
                \item Richer representation
                \item Better pattern recognition
            \end{itemize}
        \end{column}
        
        \begin{column}{0.48\textwidth}
            \textbf{Backward LSTM:}
            \begin{itemize}
                \item Reads $t \to t-1 \to t-60$
                \item Captures structure
                \item Sees overall pattern
            \end{itemize}
            
            \vspace{0.2cm}
            
            \textbf{No Lookahead Bias:}
            \begin{itemize}
                \item Only uses $t-60$ to $t$
                \item Target is $t+1$ to $t+h$
                \item Strict temporal separation
            \end{itemize}
        \end{column}
    \end{columns}
\end{frame}

\begin{frame}{Training Protocol: Preventing Overfitting}
        \begin{itemize}
            \item \textbf{Train}: 1993-2015 (5,744 days) - Model parameter estimation
            \item \textbf{Validation}: 2016-2019 (1,006 days) - Early stopping \& hyperparameter tuning
            \item \textbf{Test}: 2020-2025 (1,455 days) - Truly out-of-sample evaluation
        \end{itemize}
    
    \vspace{0.2cm}
    
    \begin{columns}[T]
        \begin{column}{0.48\textwidth}
            \textbf{Regularization:}
            \begin{itemize}
                \item Dropout: 0.2
                \item L2 weight decay
                \item Early stopping (patience=10)
                \item Batch normalization
            \end{itemize}
        \end{column}
        
        \begin{column}{0.48\textwidth}
            \textbf{Optimization:}
            \begin{itemize}
                \item Adam optimizer
                \item Initial LR: 0.001
                \item ReduceLROnPlateau scheduler
                \item Batch size: 32
            \end{itemize}
        \end{column}
    \end{columns}
\end{frame}

% ================================================================
% SECTION 5: ENSEMBLE & VALIDATION
% ================================================================

\section{Ensemble \& Validation}

\begin{frame}{The Ensemble Philosophy: Diverse Perspectives}
    Each model has different "blind spots" - combining them creates robustness
    
    \vspace{0.2cm}
    
    \begin{tabular}{p{2.5cm} p{4.5cm} p{4.5cm}}
        \textbf{Model} & \textbf{Strength} & \textbf{Weakness} \\
        \hline
        \textbf{GARCH} & Theoretically grounded, captures leverage effect, stable & Rigid parametric form, slow to adapt to regime changes \\[0.3cm]
        \textbf{LSTM-RV} & Flexible non-linear patterns, learns from history & Backward-looking only, no forward market info \\[0.3cm]
        \textbf{LSTM-VIX} & Forward-looking market expectations, captures sentiment & Biased by risk premium, can overreact to fear \\
    \end{tabular}
    
    \vspace{0.2cm}
    
    \begin{center}
        \textcolor{hkustblue}{\textbf{Ensemble = Best of all worlds}}
    \end{center}
\end{frame}

\begin{frame}{Multi-Horizon Forecasting: Why Different Horizons?}
    \begin{columns}[T]
        \begin{column}{0.41\textwidth}
            \textbf{h=2 days:}
            \begin{itemize}
                \item Highly volatile
                \item Hardest to predict
            \end{itemize}
            
            \vspace{0.05cm}
            
            \textbf{\textcolor{cyan}{h=5 days:}}
            \begin{itemize}
                \item One trading week
                \item Moderate predictability
            \end{itemize}
            
            \vspace{0.05cm}
            
            \textbf{h=10, \textcolor{orange}{h=30 days:}}
            \begin{itemize}
                \item Smoother trajectories
                \item Easier to forecast
                \item h=30 matches VIX
            \end{itemize}
            
            \vspace{0.05cm}
            
            \textcolor{hkustblue}{\textbf{Longer horizons $\Rightarrow$ lower noise}}
        \end{column}
        
        \begin{column}{0.57\textwidth}
            \begin{figure}
                \centering
                \includegraphics[width=1\textwidth]{pic/07_rv_horizons.png}
            \end{figure}
        \end{column}
    \end{columns}
\end{frame}

% --- INSERTED: GARCH multi-step forecasting slide ---
\begin{frame}{GARCH: Multi-Step Forecasting}
    \textbf{Iterate the variance recursion:}
    $$\sigma_{t+1|t}^2 = \omega + \alpha_1 r_t^2 + \alpha_2 r_{t-1}^2 + \beta_1 \sigma_t^2$$
    $$\sigma_{t+k|t}^2 = \omega + (\alpha_1+\beta_1)\sigma_{t+k-1|t}^2 + \alpha_2\sigma_{t+k-2|t}^2 \quad (k \geq 2)$$
    
    \vspace{0.3cm}
    
    \textbf{Aggregate to forward $h$-day RV:}
    $$\boxed{\widehat{RV}^{fwd}_h(t) = \sqrt{\frac{252}{h}\sum_{i=1}^{h} \sigma_{t+i|t}^2}}$$
    
    \vspace{0.3cm}
    
    \begin{center}
        \textcolor{hkustblue}{\textbf{Sum future variances $\rightarrow$ average $\rightarrow$ annualize}}
    \end{center}
\end{frame}

% --- INSERTED: LSTM multi-horizon slide ---
\begin{frame}{LSTMs: Direct Multi-Horizon Prediction}
    \textbf{No rollout needed — predict $RV^{fwd}_h$ directly:}
    $$\widehat{RV}^{fwd}_h = \exp\Big(f_\theta(X_{t-59:t})\Big)$$
    
    \vspace{0.3cm}
    
    \begin{columns}[T]
        \begin{column}{0.48\textwidth}
            \textbf{GARCH approach:}
            \begin{itemize}
                \item Iterate $\sigma^2_{t+1}, \ldots, \sigma^2_{t+h}$
                \item Aggregate variances
                \item Error compounds over steps
            \end{itemize}
        \end{column}
        \begin{column}{0.48\textwidth}
            \textbf{LSTM approach:}
            \begin{itemize}
                \item One-shot prediction for each $h$
                \item Learns $h$-specific patterns
                \item No compounding error
            \end{itemize}
        \end{column}
    \end{columns}
    
    \vspace{0.3cm}
    
    \begin{center}
        \textcolor{hkustblue}{\textbf{Train separate model per horizon}}
    \end{center}
\end{frame}



\begin{frame}{Ensemble Weight Optimization}
        For each horizon h $\in$ \{2, 5, 10, 30\}:
        \begin{enumerate}
            \item Generate predictions from all three models on \textbf{validation set}
            \item Grid search over weight combinations: $w_g, w_{lrv}, w_{lvix}$ where $w_g + w_{lrv} + w_{lvix} = 1$
            \item Combine in log-space: $\log(\widehat{RV}_{ens}) = w_g \log(\widehat{RV}_g) + w_{lrv} \log(\widehat{RV}_{lrv}) + w_{lvix} \log(\widehat{RV}_{lvix})$
            \item Select weights that minimize validation RMSE
        \end{enumerate}
    
    \vspace{0.2cm}
    
        \begin{tabular}{c c c c c}
            \textbf{Horizon} & \textbf{GARCH} & \textbf{LSTM-RV} & \textbf{LSTM-VIX} & \textbf{Val RMSE} \\
            \hline
            h=2 & 0.50 & 0.00 & 0.50 & 0.0844 \\
            h=5 & 0.30 & 0.00 & 0.70 & 0.0599 \\
            h=10 & 0.30 & 0.00 & 0.70 & 0.0547 \\
            h=30 & 0.20 & 0.00 & 0.80 & 0.0493 \\
        \end{tabular}
\end{frame}

\begin{frame}{Surprising Finding: LSTM-RV Gets Zero Weight}
        Despite being a sophisticated neural network trained on historical volatility, LSTM-RV receives \textbf{0\% weight} in all four optimal ensembles (h=2,5,10,30)
    
    \vspace{0.2cm}
    
    \begin{center}
        \textbf{Why would a trained model be completely ignored?}
    \end{center}
\end{frame}

\begin{frame}{Explanation 1: Redundancy with GARCH}
        Both LSTM-RV and GARCH learn from \textbf{historical realized volatility}
    
    \vspace{0.2cm}
    
    \begin{columns}[T]
        \begin{column}{0.48\textwidth}
            \textbf{GARCH's Advantage:}
            \begin{itemize}
                \item Explicit autoregressive formula
                \item Mean reversion built-in
                \item Leverage effect via EGARCH
                \item Theoretically grounded
                \item Converges to long-run variance
                \item 30+ years of econometric research
            \end{itemize}
        \end{column}
        
        \begin{column}{0.48\textwidth}
            \textbf{LSTM-RV's Challenge:}
            \begin{itemize}
                \item Learns patterns from data
                \item Flexible but opaque
                \item Can overfit to training regimes
                \item No built-in mean reversion
                \item May not generalize to 2016-2019
                \item Needs more data than available
            \end{itemize}
        \end{column}
    \end{columns}
    
    \vspace{0.2cm}
    
    \begin{center}
        \textcolor{hkustblue}{\textbf{Result: GARCH captures the same information more efficiently}}
    \end{center}
\end{frame}

\begin{frame}{Explanation 2: VIX Information Dominates}
        VIX contains market's collective wisdom about future volatility
    
    \vspace{0.2cm}
    
    \textbf{Information Content Comparison:}
    
    \vspace{0.2cm}
    
    \begin{tabular}{p{3cm} p{5cm} p{3cm}}
        \textbf{Source} & \textbf{Information} & \textbf{R$^2$ with RV} \\
        \hline
        Historical RV & What volatility \textit{was} & 0.42 \\[0.2cm]
        VIX & What market \textit{expects} & 0.67 \\[0.2cm]
    \end{tabular}
    
    \vspace{0.2cm}
    
    \begin{itemize}
        \item VIX explains 67\% of future RV variance
        \item Historical RV only explains 42\%
        \item VIX incorporates: options flow, sentiment, macro events, institutional positioning
        \item Historical RV is backward-looking only
    \end{itemize}
    
    \vspace{0.2cm}
    
    \begin{center}
        \textcolor{hkustred}{\textbf{Once you have VIX, historical RV adds little new information}}
    \end{center}
\end{frame}


\begin{frame}{Synthesis: Why Zero Weight Makes Sense}
    \begin{center}
        \textbf{Bringing It All Together}
    \end{center}
    
    \vspace{0.2cm}
    
    \begin{enumerate}
        \item \textbf{Redundancy}: GARCH captures autoregressive patterns more efficiently than LSTM-RV
        
        \vspace{0.2cm}
        
        \item \textbf{Information hierarchy}: VIX (forward-looking) $>$ Historical RV (backward-looking)
        
        \vspace{0.2cm}
        
        \item \textbf{Generalization}: LSTM-RV may overfit to 1993-2015 training regime
        
        \vspace{0.2cm}
        
        \item \textbf{Ensemble theory}: Need \textit{complementary} models, not similar ones
    \end{enumerate}
    
    \vspace{0.2cm}
    
        The optimal ensemble combines:
        \begin{itemize}
            \item \textbf{GARCH}: Econometric structure from past volatility (20-50\% weight)
            \item \textbf{LSTM-VIX}: Neural network de-biasing of market expectations (50-80\% weight)
        \end{itemize}
        
        This pairing provides both theoretical grounding \textit{and} forward-looking information without redundancy.
\end{frame}

\begin{frame}{Validation Performance: RMSE Across Horizons}
        Validation RMSE (2016-2019) — Lower is better
    
    \vspace{0.2cm}
    
    \begin{center}
        \begin{tabular}{c c c c c}
            \textbf{Horizon} & \textbf{GARCH} & \textbf{LSTM-RV} & \textbf{LSTM-VIX} & \textbf{Ensemble} \\
            \hline
            h=2 & 0.0968 & 0.0960 & 0.0928 & \textcolor{hkustblue}{\textbf{0.0912}} \\
            h=5 & 0.0686 & 0.0679 & 0.0651 & \textcolor{hkustblue}{\textbf{0.0638}} \\
            h=10 & 0.0634 & 0.0632 & 0.0602 & \textcolor{hkustblue}{\textbf{0.0589}} \\
            h=30 & 0.0601 & 0.0598 & 0.0564 & \textcolor{hkustblue}{\textbf{0.0551}} \\
        \end{tabular}
    \end{center}
    
    \vspace{0.2cm}
    
    \textbf{Key Observations:}
    \begin{itemize}
        \item Ensemble \textbf{always} outperforms individual models
        \item LSTM-VIX consistently best single model
        \item \textbf{Lower RMSE for longer horizons} — targets are smoother (less volatile)
        \item Absolute error decreases, but relative error stays similar
        \item h=30 is genuinely easier to predict than h=2 in absolute terms
    \end{itemize}
\end{frame}

\begin{frame}{Understanding the Horizon Effect: Why RMSE Decreases}
        RMSE decreases as horizon increases — longer horizons ARE easier in absolute terms:
    
    \vspace{0.2cm}
    
    \begin{columns}[T]
        \begin{column}{0.48\textwidth}
            \textbf{The Time Averaging Effect:}
            \begin{itemize}
                \item h=2: volatility of 2 days (noisy)
                \item h=30: volatility averaged over 30 days (smooth)
                \item Longer horizons $\Rightarrow$ less volatile targets
                \item Lower target variability $\Rightarrow$ lower absolute errors
            \end{itemize}
            
            \vspace{0.2cm}
            
            $$\text{Var}(\bar{r}_{30}) < \text{Var}(\bar{r}_2)$$
        \end{column}
        
        \begin{column}{0.48\textwidth}
            \textbf{GARCH Mean Reversion:}
            \begin{itemize}
                \item Short-term: sensitive to recent shocks
                \item Long-term: forecasts converge to unconditional mean
                \item h=30 prediction $\approx$ "volatility will be normal"
                \item Safe but uninformative!
            \end{itemize}
            
            \vspace{0.2cm}
            
            $$\sigma^2_{t+k} \to \frac{\omega}{1-\alpha-\beta}$$
        \end{column}
    \end{columns}
    
    \vspace{0.2cm}
    
    \begin{center}
        \textcolor{hkustblue}{\textbf{Longer horizons: lower absolute error, similar relative error}}
    \end{center}
\end{frame}


% ================================================================
% SECTION 6: TRADING STRATEGIES
% ================================================================

\section{Trading Strategies}

\begin{frame}{The Central Question}
    \begin{center}
        \Large\textit{Where do our volatility predictions actually create value?}
    \end{center}
    
    \vspace{0.4cm}
    
    \textbf{Two Hypothesis Families:}
    \begin{enumerate}
        \item \textbf{Equity Trading}: Can predictions improve position sizing?
        \item \textbf{Volatility Trading}: Can predictions identify VRP mispricings?
    \end{enumerate}
    
    \vspace{0.3cm}
    
    \textbf{Test Period}: 2020-2025 (out-of-sample)
    \begin{itemize}
        \item COVID crash, recovery, 2022 volatility, recent markets
        \item Expanding windows for all thresholds (no lookahead bias)
    \end{itemize}
\end{frame}

\begin{frame}{Strategy Overview: Seven Approaches Tested}
    \textbf{Family 1: Equity Strategies}
    \begin{enumerate}
        \item SPY Buy \& Hold — Passive benchmark
        \item SPY SMA(50) Trend — Avoid bear markets
        \item SPY Trend + VIX Sizing — Scale by VIX percentile
        \item SPY Trend + Prediction Sizing — Scale by our forecast
    \end{enumerate}
    
    \vspace{0.3cm}
    
    \textbf{Family 2: Volatility Strategies}
    \begin{enumerate}
        \setcounter{enumi}{4}
        \item VRP Unconditional — Always sell volatility
        \item VRP VIX-Based — Sell when VIX is high
        \item \textcolor{hkustred}{\textbf{VRP Residual-Based}} — Sell when VIX is \textit{abnormally} high
    \end{enumerate}
\end{frame}

\begin{frame}{The Volatility Risk Premium: VIX vs Realized}
    \begin{columns}[T]
        \begin{column}{0.48\textwidth}
            \textbf{Three Time Series:}
            \begin{itemize}
                \item \textcolor{cyan}{Blue}: VIX (expectations)
                \item \textcolor{purple}{Red}: Actual RV
                \item \textcolor{LimeGreen}{Green}: Positive VIX - RV spread
            \end{itemize}
            
            \vspace{0.2cm}
            
            \textbf{Key Patterns:}
            \begin{itemize}
                \item VIX $>$ RV most of the time
                \item Average spread: $\sim$3-5 vol points
                \item Widens in crises, occasionally inverts
            \end{itemize}
            
            \vspace{0.2cm}
            
            \textbf{Trading Opportunity:}
            \begin{itemize}
                \item Investors overpay for protection
                \item Insurance premium = systematic profit
                \item \textbf{This is what we trade!}
            \end{itemize}
        \end{column}
        
        \begin{column}{0.5\textwidth}
            \begin{figure}
                \centering
                \includegraphics[width=\textwidth]{pic/06_rv_vs_vix.png}
            \end{figure}
        \end{column}
    \end{columns}
\end{frame}

\begin{frame}{Position Sizing: VIX vs Prediction}
    \textbf{Core Idea:} Use more leverage when volatility is low, less when high
    
    \vspace{0.3cm}
    
    \textbf{Step 1: Trend Filter}
    $$\text{Trend} = \begin{cases} 1 & \text{if } \text{SPY}_t > \text{SMA}_{50}(t-1) \\ 0 & \text{otherwise (stay in cash)} \end{cases}$$
    
    \vspace{0.2cm}
    
    \textbf{Step 2: Position Multiplier} (rolling 252-day percentile)
    $$\text{Multiplier} = 1.0 + 0.5 \times (1 - \text{Percentile}_{t-1})$$
    
    \begin{itemize}
        \item Low volatility (0th pctl): Multiplier = \textbf{1.5$\times$} (use leverage)
        \item High volatility (100th pctl): Multiplier = \textbf{1.0$\times$} (no leverage)
    \end{itemize}
    
    \vspace{0.2cm}
    
    \textbf{VIX Sizing:} Percentile of VIX $\quad$ vs $\quad$ \textbf{Pred Sizing:} Percentile of $\widehat{RV}_{ensemble}$
\end{frame}

\begin{frame}{Equity Strategies: Results Summary}
    \begin{center}
        \small
        \begin{tabular}{l c c c c}
            \textbf{Strategy} & \textbf{Return} & \textbf{Sharpe} & \textbf{Max DD} & \textbf{Bear Mkt} \\
            \hline
            1. Buy \& Hold & 12.5\% & 0.60 & -35.7\% & -44.2\% \\
            2. SMA Trend & 10.3\% & \textbf{0.89} & -21.2\% & -32.0\% \\
            3. Trend + VIX & 13.6\% & 0.91 & -27.1\% & -38.2\% \\
            4. Trend + Pred & 13.2\% & 0.87 & -27.1\% & -38.5\% \\
        \end{tabular}
    \end{center}
    
    \vspace{0.3cm}
    
    \textbf{Key Finding:}
    \begin{itemize}
        \item Prediction-based sizing performs \textit{identically} to VIX-based sizing
        \item Correlation between predictions and VIX: $\rho \approx 0.95$
        \item \textcolor{hkustred}{\textbf{No incremental value for equity position sizing}}
    \end{itemize}
\end{frame}

\begin{frame}{Why Predictions Don't Help Equity Trading}
    \textbf{The Problem:}
    \begin{itemize}
        \item VIX already provides real-time risk information
        \item Our predictions are highly correlated with VIX ($\rho \approx 0.95$)
        \item For position sizing, we need a risk indicator — VIX suffices
    \end{itemize}
    
    \vspace{0.3cm}
    
    \textbf{The Insight:}
    \begin{itemize}
        \item Prediction accuracy $\neq$ Signal value
        \item Value comes from information \textit{the market doesn't have}
        \item Our predictions replicate what VIX already tells us
    \end{itemize}
    
    \vspace{0.3cm}
    
    \begin{center}
        \textcolor{hkustblue}{\textbf{Pivot: What do our predictions tell us that VIX doesn't?}}
    \end{center}
\end{frame}

\begin{frame}{The Volatility Risk Premium (VRP)}
    $$\text{VRP} = \text{VIX} - \text{Realized Volatility}$$
    
    \vspace{0.2cm}
    
    \textbf{The Trade:} Sell variance swap, profit = $(IV^2 - RV^2) \times \text{Notional}$
    
    \vspace{0.3cm}
    
    \textbf{Historical Statistics (2020-2025):}
    \begin{itemize}
        \item VRP positive: \textbf{83.6\%} of 30-day periods
        \item Mean P\&L: +0.79\% per period
        \item Std Dev: 9.69\% per period
    \end{itemize}
    
    \vspace{0.3cm}
    
    \textbf{The Challenge:}
    \begin{itemize}
        \item Premium exists but tail risk is catastrophic
        \item How to harvest premium while avoiding crashes?
    \end{itemize}
\end{frame}

\begin{frame}{Strategy 5: VRP Unconditional — Always Sell}
    \textbf{Implementation:} Sell variance swap every 30-day period
    
    \vspace{0.3cm}
    
    \begin{columns}[T]
        \begin{column}{0.48\textwidth}
            \textbf{The Good:}
            \begin{itemize}
                \item Return: \textbf{51.8\%} annualized
                \item Sharpe: \textbf{1.86}
                \item Bull Market: +244\% ann.
            \end{itemize}
        \end{column}
        
        \begin{column}{0.48\textwidth}
            \textbf{The Bad:}
            \begin{itemize}
                \item Max DD: \textcolor{hkustred}{\textbf{-96.4\%}}
                \item Bear Market: -74.7\% ann.
                \item Near total wipeout in crashes
            \end{itemize}
        \end{column}
    \end{columns}
    
    \vspace{0.4cm}
    
    \begin{center}
        \textcolor{hkustred}{\textbf{Not investable without filtering}}
    \end{center}
\end{frame}

\begin{frame}{Strategy 6: VRP VIX-Based — Sell When VIX High}
    \textbf{Logic:} High VIX = larger premium, mean reversion expected
    
    \textbf{Implementation:} Sell only when VIX $>$ 70th percentile
    
    \vspace{0.3cm}
    
    \begin{columns}[T]
        \begin{column}{0.48\textwidth}
            \textbf{Results:}
            \begin{itemize}
                \item Return: 9.9\% ann.
                \item Sharpe: 1.22
                \item Max DD: -58.2\%
                \item Trades: 27\% of periods
            \end{itemize}
        \end{column}
        
        \begin{column}{0.48\textwidth}
            \textbf{Problem:}
            \begin{itemize}
                \item High VIX $\neq$ Safe to sell
                \item Can't distinguish:
                \begin{itemize}
                    \item Justified fear (crisis)
                    \item Excessive fear (opportunity)
                \end{itemize}
            \end{itemize}
        \end{column}
    \end{columns}
    
    \vspace{0.3cm}
    
    \begin{center}
        \textcolor{hkustblue}{\textbf{Need: independent forecast to judge if VIX is "too high"}}
    \end{center}
\end{frame}

\begin{frame}{Why We Need Expected VRP (Not Just Raw VRP)}
    \textbf{The Problem with Raw VRP:}
    $$\text{VRP}_{forecast} = \text{VIX} - \widehat{RV}_{ensemble}$$
    
    VRP naturally scales with VIX level:
    \begin{itemize}
        \item VIX = 40 $\Rightarrow$ typical VRP $\approx$ 8-10 points
        \item VIX = 15 $\Rightarrow$ typical VRP $\approx$ 2-3 points
    \end{itemize}
    
    \vspace{0.2cm}
    
    \textbf{A VRP of 8 means different things:}
    \begin{itemize}
        \item At VIX = 40: \textbf{Normal} (expected given high fear)
        \item At VIX = 20: \textbf{Unusually large} (fear exceeds fundamentals)
    \end{itemize}
    
    \vspace{0.2cm}
    
    \textbf{Solution:} Model the VIX-VRP relationship, trade the \textit{deviation}
    $$\text{Expected VRP} = \alpha + \beta \times \text{VIX}$$
    $$\textbf{Residual} = \text{VRP}_{forecast} - \text{Expected VRP}$$
\end{frame}

\begin{frame}{Strategy 7: VRP Residual-Based — Our Edge}
    \textbf{Key Innovation:} Compare VIX to our \textit{independent} RV forecast
    
    \vspace{0.2cm}
    
    $$\text{VRP}_{forecast} = \text{VIX} - \widehat{RV}_{ensemble}$$
    $$\text{Expected VRP} = \alpha + \beta \times \text{VIX} \quad \text{(fitted on history)}$$
    $$\textbf{Residual} = \text{VRP}_{forecast} - \text{Expected VRP}$$
    
    \vspace{0.2cm}
    
    \textbf{Trading Rule:} Sell when Residual $>$ 70th percentile
    
    \vspace{0.2cm}
    
    \textbf{Interpretation:}
    \begin{itemize}
        \item High residual $\Rightarrow$ VIX is \textit{abnormally} high given our RV forecast
        \item Market fear exceeds what fundamentals justify
        \item \textcolor{hkustred}{\textbf{This is mispricing we can exploit}}
    \end{itemize}
\end{frame}

\begin{frame}{Residual-Based Strategy: Worked Examples}
    \textbf{Example 1: COVID Crash (Don't Sell)}
    \begin{itemize}
        \item VIX = 70, Our forecast: RV = 65
        \item VRP$_{forecast}$ = 5, Expected VRP at VIX=70: $\sim$8
        \item Residual = 5 - 8 = \textbf{-3} $\Rightarrow$ \textcolor{hkustblue}{Stay out}
    \end{itemize}
    
    \vspace{0.3cm}
    
    \textbf{Example 2: Fear Spike (Sell)}
    \begin{itemize}
        \item VIX = 35, Our forecast: RV = 18
        \item VRP$_{forecast}$ = 17, Expected VRP at VIX=35: $\sim$9
        \item Residual = 17 - 9 = \textbf{+8} $\Rightarrow$ \textcolor{hkustred}{Sell volatility}
    \end{itemize}
    
    \vspace{0.3cm}
    
    \begin{center}
        \textbf{The residual isolates \textit{unjustified} fear from \textit{justified} fear}
    \end{center}
\end{frame}

\begin{frame}{VRP Strategies: Head-to-Head Comparison}
    \begin{center}
        \small
        \begin{tabular}{l c c c c}
            \textbf{Strategy} & \textbf{Sharpe} & \textbf{Return} & \textbf{Max DD} & \textbf{Bear Mkt} \\
            \hline
            5. Unconditional & 1.86 & 51.8\% & -96.4\% & -74.7\% \\
            6. VIX-Based & 1.22 & 9.9\% & -58.2\% & -2.7\% \\
            \textbf{7. Residual-Based} & \textcolor{hkustred}{\textbf{4.97}} & \textbf{25.4\%} & \textbf{-27.7\%} & \textcolor{hkustred}{\textbf{+11.3\%}} \\
        \end{tabular}
    \end{center}
    
    \vspace{0.3cm}
    
    \textbf{Residual-Based Advantages:}
    \begin{itemize}
        \item Sharpe 4.97 — nearly 3$\times$ Unconditional, 4$\times$ VIX-Based
        \item \textbf{Positive bear market returns} (+11.3\% when SPY loses -44\%)
        \item Beta = 0.013 — true diversification from equities
        \item Max DD -27.7\% — investable risk level
    \end{itemize}
\end{frame}

\begin{frame}{Robustness Across All Horizons}
    \begin{center}
        \begin{tabular}{c c c c c}
            \textbf{Horizon} & \textbf{Sharpe} & \textbf{Ann. Return} & \textbf{Max DD} & \textbf{Bear Mkt} \\
            \hline
            h=2 & 4.39 & 23.9\% & -27.7\% & +3.2\% \\
            \textbf{h=5} & \textcolor{hkustred}{\textbf{5.94}} & \textbf{31.6\%} & -27.7\% & +14.1\% \\
            h=10 & 4.84 & 24.8\% & -27.7\% & +11.3\% \\
            h=30 & 4.97 & 25.4\% & -27.7\% & +11.3\% \\
        \end{tabular}
    \end{center}
    
    \vspace{0.3cm}
    
    \textbf{Key Observations:}
    \begin{itemize}
        \item Strategy works across \textit{all} forecast horizons
        \item h=5 (one trading week) performs best: Sharpe 5.94
        \item Consistent drawdown control across horizons
        \item \textbf{Not overfitting to a single time scale}
    \end{itemize}
\end{frame}

\begin{frame}{Final Rankings: All Seven Strategies (h=30)}
    \begin{center}
        \scriptsize
        \begin{tabular}{c l c c c c c c}
            \textbf{Rank} & \textbf{Strategy} & \textbf{Sharpe} & \textbf{Return} & \textbf{Max DD} & \textbf{Alpha} & \textbf{Beta} & \textbf{Bear Mkt} \\
            \hline
            1 & \textbf{VRP Residual} & \textcolor{hkustred}{\textbf{4.97}} & 25.4\% & -27.7\% & \textbf{0.23} & \textbf{0.01} & \textcolor{hkustred}{+11.3\%} \\
            2 & VRP Unconditional & 1.86 & 51.8\% & -96.4\% & 0.43 & 0.18 & -74.7\% \\
            3 & VRP VIX-Based & 1.22 & 9.9\% & -58.2\% & 0.10 & 0.00 & -2.7\% \\
            4 & Trend + VIX & 0.91 & 13.6\% & -27.1\% & 0.08 & 0.39 & -38.2\% \\
            5 & SMA Trend & 0.89 & 10.3\% & -21.2\% & 0.06 & 0.30 & -32.0\% \\
            6 & Trend + Pred & 0.87 & 13.2\% & -27.1\% & 0.08 & 0.40 & -38.5\% \\
            7 & Buy \& Hold & 0.60 & 12.5\% & -35.7\% & 0.00 & 1.00 & -44.2\% \\
        \end{tabular}
    \end{center}
    
    \vspace{0.2cm}
    
    \textbf{VRP Residual: highest alpha (0.23), lowest beta (0.01), only positive bear returns}
\end{frame}

\begin{frame}{Limitations \& Practical Considerations}
    \textbf{Not Included in Backtest:}
    \begin{itemize}
        \item Transaction costs (bid-ask spreads on variance swaps)
        \item Slippage and market impact
        \item Margin requirements and financing costs
        \item Roll costs for VIX futures implementation
    \end{itemize}
    
    \vspace{0.3cm}
    
    \textbf{Variance Swap Modeling:}
    \begin{itemize}
        \item Theoretical P\&L: $\text{VIX}^2 - \text{RV}_{30}^2$ (no actual market prices)
        \item Assumes execution at fair value (VIX strike)
    \end{itemize}
    
    \vspace{0.3cm}
    
    \textbf{Realistic Expectation:}
    \begin{itemize}
        \item Sharpe ratios will degrade with costs (estimate 20-40\% reduction)
        \item Core insight remains: residual-based filtering outperforms
    \end{itemize}
\end{frame}

% --- Thank you slide ---
\begin{frame}
\begin{center}
{ Thank you for listening !}
\vspace{1cm}

Francesco, Nikhil, Ivan \\[1em]
\end{center}
\end{frame}

% ================================================================
% APPENDIX
% ================================================================

\appendix

\section{Appendix}

\begin{frame}{Appendix: GARCH Model Selection \& Diagnostics}
    \textbf{Best Model:} EGARCH(2,1) with Skewed Student-t
    
    \vspace{0.2cm}
    
    \begin{columns}[T]
        \begin{column}{0.48\textwidth}
            \textbf{Pre-Fit Tests:}
            \scriptsize
            \begin{tabular}{l r}
                ADF p-value & $2.1 \times 10^{-25}$ \\
                KPSS p-value & 0.100 \\
                ARCH-LM p-value & $2.2 \times 10^{-292}$ \\
                Jarque-Bera p-value & 0.000 \\
            \end{tabular}
            
            \vspace{0.2cm}
            \normalsize
            \textbf{Post-Fit Tests:}
            \scriptsize
            \begin{tabular}{l r}
                ARCH-LM p-value & 0.839 \\
                ARCH Pass & \checkmark \\
                Ljung-Box Pass & \checkmark \\
            \end{tabular}
        \end{column}
        
        \begin{column}{0.48\textwidth}
            \textbf{EGARCH Parameters:}
            \scriptsize
            \begin{tabular}{l r}
                $\mu$ & 0.0223 \\
                $\omega$ & 0.0036 \\
                $\alpha_1$ & -0.0496 \\
                $\alpha_2$ & 0.2079 \\
                $\gamma_1$ (leverage) & \textbf{-0.1633} \\
                $\beta_1$ & 0.9724 \\
                $\eta$ (d.o.f.) & 7.997 \\
                $\lambda$ (skew) & -0.135 \\
            \end{tabular}
            
            \vspace{0.1cm}
            \normalsize
            BIC: 15,401.93
        \end{column}
    \end{columns}
    
    \vspace{0.2cm}
    
    \scriptsize
    Training Period: 1993-03-15 to 2015-12-31
\end{frame}

\begin{frame}{Appendix: Model Performance — RMSE by Horizon}
    \textbf{Root Mean Square Error (lower is better)}
    
    \vspace{0.3cm}
    
    \begin{center}
        \small
        \begin{tabular}{l | c c c | c c c | c c c}
            & \multicolumn{3}{c|}{\textbf{GARCH}} & \multicolumn{3}{c|}{\textbf{LSTM-RV}} & \multicolumn{3}{c}{\textbf{LSTM-VIX}} \\
            \textbf{h} & Train & Val & Test & Train & Val & Test & Train & Val & Test \\
            \hline
            2 & 0.129 & 0.097 & 0.149 & 0.139 & 0.096 & 0.129 & 0.129 & \textbf{0.090} & \textbf{0.120} \\
            5 & 0.083 & 0.069 & 0.107 & 0.086 & 0.069 & 0.092 & 0.080 & \textbf{0.061} & \textbf{0.080} \\
            10 & 0.070 & 0.063 & 0.102 & 0.073 & 0.062 & 0.078 & 0.068 & \textbf{0.056} & \textbf{0.071} \\
            30 & 0.069 & 0.060 & 0.113 & 0.070 & 0.058 & 0.072 & 0.067 & \textbf{0.050} & \textbf{0.063} \\
        \end{tabular}
    \end{center}
    
    \vspace{0.3cm}
    
    \textbf{Key Observations:}
    \begin{itemize}
        \item LSTM-VIX achieves lowest validation and test RMSE across all horizons
        \item GARCH shows higher test RMSE (potential overfitting to training regime)
        \item Longer horizons have lower absolute RMSE (smoother targets)
    \end{itemize}
\end{frame}

\begin{frame}{Appendix: Model Performance — MAE by Horizon}
    \textbf{Mean Absolute Error (lower is better)}
    
    \vspace{0.3cm}
    
    \begin{center}
        \small
        \begin{tabular}{l | c c c | c c c | c c c}
            & \multicolumn{3}{c|}{\textbf{GARCH}} & \multicolumn{3}{c|}{\textbf{LSTM-RV}} & \multicolumn{3}{c}{\textbf{LSTM-VIX}} \\
            \textbf{h} & Train & Val & Test & Train & Val & Test & Train & Val & Test \\
            \hline
            2 & 0.094 & 0.073 & 0.101 & 0.085 & 0.057 & 0.082 & 0.082 & \textbf{0.055} & \textbf{0.079} \\
            5 & 0.057 & 0.052 & 0.068 & 0.056 & 0.045 & 0.060 & 0.052 & \textbf{0.040} & \textbf{0.055} \\
            10 & 0.048 & 0.049 & 0.063 & 0.046 & 0.045 & 0.051 & 0.043 & \textbf{0.037} & \textbf{0.047} \\
            30 & 0.046 & 0.050 & 0.063 & 0.041 & 0.046 & 0.048 & 0.039 & \textbf{0.036} & \textbf{0.042} \\
        \end{tabular}
    \end{center}
    
    \vspace{0.3cm}
    
    \textbf{Key Observations:}
    \begin{itemize}
        \item MAE confirms LSTM-VIX superiority
        \item MAE less sensitive to outliers than RMSE
        \item Consistent ranking across all horizons
    \end{itemize}
\end{frame}

\begin{frame}{Appendix: Ensemble Weights by Horizon}
    \textbf{Optimized on Validation Set (2016-2019)}
    
    \vspace{0.3cm}
    
    \begin{center}
        \begin{tabular}{c c c c}
            \textbf{Horizon} & \textbf{GARCH} & \textbf{LSTM-RV} & \textbf{LSTM-VIX} \\
            \hline
            h=2 & 0.50 & 0.00 & 0.50 \\
            h=5 & 0.30 & 0.00 & 0.70 \\
            h=10 & 0.30 & 0.00 & 0.70 \\
            h=30 & 0.20 & 0.00 & 0.80 \\
        \end{tabular}
    \end{center}
    
    \vspace{0.3cm}
    
    \textbf{Pattern:}
    \begin{itemize}
        \item LSTM-RV receives \textbf{zero weight} at all horizons
        \item LSTM-VIX dominates (50-80\% weight)
        \item GARCH weight decreases with horizon (50\% → 20\%)
        \item Longer horizons favor forward-looking VIX information
    \end{itemize}
\end{frame}

\begin{frame}{Appendix: Strategy Metrics — All Horizons (1/2)}
    \textbf{Equity Strategies (h=30 shown, identical across horizons)}
    
    \vspace{0.2cm}
    
    \begin{center}
        \scriptsize
        \begin{tabular}{l c c c c c c}
            \textbf{Strategy} & \textbf{Total Ret} & \textbf{Ann. Ret} & \textbf{Vol} & \textbf{Sharpe} & \textbf{Max DD} & \textbf{Beta} \\
            \hline
            Buy \& Hold & 97.7\% & 12.5\% & 21.0\% & 0.60 & -35.7\% & 1.00 \\
            SMA Trend & 76.5\% & 10.3\% & 11.6\% & 0.89 & -21.2\% & 0.30 \\
            Trend + VIX & 109.1\% & 13.6\% & 15.1\% & 0.91 & -27.1\% & 0.39 \\
            Trend + Pred & 104.8\% & 13.2\% & 15.2\% & 0.87 & -27.1\% & 0.40 \\
        \end{tabular}
    \end{center}
    
    \vspace{0.3cm}
    
    \textbf{VRP Unconditional \& VIX-Based (identical across horizons)}
    
    \vspace{0.2cm}
    
    \begin{center}
        \scriptsize
        \begin{tabular}{l c c c c c c}
            \textbf{Strategy} & \textbf{Total Ret} & \textbf{Ann. Ret} & \textbf{Vol} & \textbf{Sharpe} & \textbf{Max DD} & \textbf{Beta} \\
            \hline
            VRP Unconditional & 1013\% & 51.8\% & 27.9\% & 1.86 & -96.4\% & 0.18 \\
            VRP VIX-Based & 72.4\% & 9.9\% & 8.1\% & 1.22 & -58.2\% & 0.00 \\
        \end{tabular}
    \end{center}
\end{frame}

\begin{frame}{Appendix: Strategy Metrics — All Horizons (2/2)}
    \textbf{VRP Residual-Based — Performance by Horizon}
    
    \vspace{0.3cm}
    
    \begin{center}
        \small
        \begin{tabular}{c c c c c c c}
            \textbf{h} & \textbf{Total Ret} & \textbf{Ann. Ret} & \textbf{Vol} & \textbf{Sharpe} & \textbf{Max DD} & \textbf{Alpha} \\
            \hline
            2 & 245\% & 23.9\% & 5.4\% & 4.39 & -27.7\% & 0.21 \\
            \textbf{5} & \textbf{388\%} & \textbf{31.6\%} & \textbf{5.3\%} & \textbf{5.94} & -27.7\% & \textbf{0.27} \\
            10 & 260\% & 24.8\% & 5.1\% & 4.84 & -27.7\% & 0.22 \\
            30 & 270\% & 25.4\% & 5.1\% & 4.97 & -27.7\% & 0.23 \\
        \end{tabular}
    \end{center}
    
    \vspace{0.3cm}
    
    \textbf{Bear vs Bull Market Returns (Residual-Based)}
    
    \vspace{0.2cm}
    
    \begin{center}
        \small
        \begin{tabular}{c c c c c}
            \textbf{h} & \textbf{Bear Ann. Ret} & \textbf{Bull Ann. Ret} & \textbf{Beta} & \textbf{Correlation} \\
            \hline
            2 & +3.2\% & +34.7\% & 0.010 & 0.038 \\
            \textbf{5} & \textbf{+14.1\%} & \textbf{+40.4\%} & 0.015 & 0.059 \\
            10 & +11.3\% & +31.6\% & 0.013 & 0.052 \\
            30 & +11.3\% & +32.5\% & 0.013 & 0.054 \\
        \end{tabular}
    \end{center}
\end{frame}

\begin{frame}{Appendix: Test Period Market Conditions}
    \textbf{Test Set: 2020-01-01 to 2025-11-27}
    
    \vspace{0.3cm}
    
    \begin{center}
        \begin{tabular}{l r}
            \textbf{Statistic} & \textbf{Value} \\
            \hline
            Total Trading Days & 1,455 \\
            Bear Market Days & 456 (31\%) \\
            Bull Market Days & 999 (69\%) \\
        \end{tabular}
    \end{center}
    
    \vspace{0.3cm}
    
    \textbf{Major Events in Test Period:}
    \begin{itemize}
        \item COVID-19 Crash (Feb-Mar 2020): VIX peaked at 82
        \item Recovery Rally (Apr 2020 - Dec 2021)
        \item 2022 Bear Market: Inflation, rate hikes
        \item 2023-2025: Mixed volatility regimes
    \end{itemize}
    
    \vspace{0.2cm}
    
    \textbf{SPY Performance:} +97.7\% total, 12.5\% annualized, -35.7\% max DD
\end{frame}

\begin{frame}{Appendix: Data Splits \& Training Details}
    \textbf{Temporal Splits:}
    
    \vspace{0.2cm}
    
    \begin{center}
        \begin{tabular}{l l r}
            \textbf{Split} & \textbf{Period} & \textbf{Days} \\
            \hline
            Training & 1993-03-15 to 2015-12-31 & 5,744 \\
            Validation & 2016-01-01 to 2019-12-31 & 1,006 \\
            Test & 2020-01-01 to 2025-11-27 & 1,455 \\
        \end{tabular}
    \end{center}
    
    \vspace{0.3cm}
    
    \textbf{LSTM Hyperparameters:}
    \begin{columns}[T]
        \begin{column}{0.48\textwidth}
            \begin{itemize}
                \item Sequence length: 60 days
                \item Hidden size: 64
                \item Layers: 2
                \item Bidirectional: Yes
            \end{itemize}
        \end{column}
        \begin{column}{0.48\textwidth}
            \begin{itemize}
                \item Dropout: 0.2
                \item Learning rate: 0.001
                \item Batch size: 32
                \item Early stopping patience: 10
            \end{itemize}
        \end{column}
    \end{columns}
    
    \vspace{0.2cm}
    
    \textbf{Lookahead Bias Prevention:} Training gap of $h \times 1.5$ calendar days
\end{frame}

\end{document}